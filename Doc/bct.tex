\documentclass{article}


\usepackage[linesnumbered,algoruled,boxed,lined]{algorithm2e}

%% Algorithmic2e 
\SetKwData{tbl}{table}
\SetKwData{newtable}{newTable}
\SetKwData{cls}{clause}
\SetKwData{newcls}{newClause}
\SetKwData{branch}{branch}
\SetKwData{newbranch}{newBranch}
\SetKwData{newbranches}{newBranches}
\SetKwData{startstep}{startStep}
\SetKwData{maxstep}{maxStep}
\SetKwData{none}{None}
\SetKwData{true}{True}
\SetKwData{false}{False}
\SetKwData{eqs}{eqs}
\SetKwData{openbranches}{openBranches}
\SetKwData{ret}{ret}

\title{Bounded Connection Tableau}


\begin{document}

\maketitle

\sections{Design Choices}
You can only work on the ``first'' branch of a table.

\section{Basic Definitions}

\paragraph{Flat equation} A \emph{flat equation} is an equation of the form $a = b$.

\paragraph{Functional Equation} A \emph{functional equation} (f-equation) is a equation of the form $f(\bar{a}) = b$.

\paragraph{Literal} A \emph{literal} is a (negated) propositional variable, or a (negated) flat equation.

\section{Complex Definitions}

\subsection{Pseudo-Literal}



\section{Flattening}
If we have the formula $\forall x . R(f(x)) = b$, we must flatten it, we therefore have $\forall x \exists c_1 (R(c_1) = b \wedge f(x) = c_1)$.


\section{Algorithms}

\subsection{Prove}
Input: Set of clauses
Output: Proof OR Invalid

\begin{algorithm}[H]

  \SetAlgoLined
  \KwIn{Table}
  
  }
  \Return \false\;

  \caption{CloseTable}
  \label{alg:close-table}
\end{algorithm}


\subsection{Closable}
\begin{algorithm}[H]

  \SetAlgoLined
  \KwIn{\branch}
  \eqs$\leftarrow$equations(\branch)
  \For{$l \in \branch$}{
    \If{complementary(head(\branch), l)}{
      \If{$unifiable_{\eqs}$(head(\branch), l)}{
        \Return \true\;
      }
    }
  }
  \Return \false\;

  \caption{Closable}
  \label{alg:close-branch-w-clause}
\end{algorithm}



\emph{Closable} checks whether a branch can be closed. The predicate
$unifiable_E(p, q)$ is true if literals $p$ and $q$ are unifiable
modulo the set of equations $E$. This is were we use our
BREU-procedure (though in the singular case, one can use simple
congruence closure).

\subsubsection{Todo}
\begin{itemize}
\item Do we need local or global closability?
\item Maybe have a congurence closure procedure here instead?
\end{itemize}


\subsection{Closing Branch w. Clause}

\begin{algorithm}[H]

  \SetAlgoLined
  \KwIn{\branch, \startstep the first step to try, $\varphi$ a clause}
  \KwResult{Result after an extension step closing \branch using $\varphi$ with index or \none}

  %% \maxstep$\leftarrow \Sigma_{\varphi \in \Gamma} (|\varphi|)$\;
  \maxstep$\leftarrow |\varphi|$\;  
  \For{$i \in [\startstep..\maxstep]$}{
    \newbranch $\leftarrow \varphi(i) :: \branch$\;
    \If{closable(\newbranch)}{
      \newcls$\leftarrow $rearrange($\varphi$, $i$)\;
      \newbranches $\leftarrow$ extend(\branch,\newcls)\;
      \Return (\newbranches, $i$)\;
    }
  }
  \Return (\none, \maxstep)\;

  \caption{Close Branch w. Clause}
  \label{alg:close-branch-w-clause}
\end{algorithm}

The purpose of \emph{Close Branch w. Clause} (see
Alg.~\ref{alg:close-branch-w-clause}) is to try and use a clause to
close a branch. The result will often be a new subtree, since the
extension step will create one branch per literal of the clause
used. If the clause is a unit-clause, a singleton branch will be
returned however.

The function call \emph{closable} can either be used in the
\emph{local} sense, where it only considers if this branch can be
closed in isolation, or in a \emph{global} sense where the current
context constrains possible substitutions.

\subsubsection{Todo}
\begin{itemize}
\item Do we need to store the substitution received from \emph{closable}.
\item Do we care about what pair of literals are closing the branch?
  Since we are using connection tableaux we can assume that the leaf
  of a branch is always used for closing a branch.
\end{itemize}


\subsection{Close Table}

\begin{algorithm}[H]

  \SetAlgoLined
  \KwIn{$\tbl$ a table, $\Gamma$ a set of clauses}
  \KwResult{$\tbl$ extended to a closed tableau using clauses $\in \Gamma$}
  \openbranches$\leftarrow$ openBranches(\tbl)\;
  \If{\openbranches = 0}{\Return \tbl}
  \branch$\leftarrow$pickBranch(\tbl)\;
  \For{\cls $\in \Gamma$}{
    \For{$i \in 1..|\cls|$}{
      $\ret \leftarrow CloseBranchWClause(\branch, \cls, $i$)$\;
      \If{$\ret \neq \none$}{
        $\newtable \leftarrow replaceBranch(\tbl, \branch, \ret)$\;
        \Return $CloseTable(\newtable, \Gamma)$\;
      }
    }
  }
  \Return \none
 \caption{Search}
\end{algorithm}

\subsubsection{Todo}
\begin{itemize}
  \item Change index for literals in clause to subroutine returning a list instead?
\end{itemize}


\end{document}
